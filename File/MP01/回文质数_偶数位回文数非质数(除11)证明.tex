\documentclass[UTF8]{ctexart}
\usepackage[fleqn]{amsmath}
\usepackage{listings}
\usepackage{xcolor}
\usepackage{url}
\usepackage{booktabs}
\usepackage{tabularx}
\usepackage{graphicx}
\usepackage{subfigure}
\usepackage{comment}
\usepackage{amsmath,amssymb}
\usepackage{cite}
\usepackage{fullpage}
\usepackage{fancyvrb}
\begin{document}
\title{回文质数关于偶数位回文数非质数(除11)的证明}
\date{\today}
\author{WinnieVenice}
\maketitle
因为一个n位数的数都可以表示成:$\sum_{i=1}^n a_i*10^{i-1}$,所以一个偶数位回文数可以表示为:$\sum_{i=1}^{2k} a_i*10^{i-1}$。
根据回文性质又可以表示为$N=\sum_{i=1}^k a_i*(10^{2k-i}+10^{i-1})$。

要证明N不是质数,只需要证明N是合数,根据合数的性质也就只需要证明$N\equiv0$ mod $\alpha,\alpha\in \textbf{常数}$。
由于N的任意性即$a_i$的任意性,所以有个显然的想法是证明$\forall (10^{2k-i}+10^{i-1})\equiv0$ mod $\alpha$。
\begin{align*}
\textbf{证明} \quad &\because  1\leq i \leq k \\
&\therefore  2k-i\geq k > k-1 \geq i-1 \geq 0 \\
&\therefore  10^{2k-i}+10^{i-1} \leftrightarrow 10^{i-1}*(10^{2*(k-i)+1}+1) \\
&\because  i=1,10^{i-1} \equiv 1 \\
&\therefore  \textbf{我们去证明:} \forall 10^{2*(k-i)+1}+1 \equiv 0 \ \textup{mod} \ \alpha 
\end{align*}
\qquad因为$11\leq10^{2*(k-i)+1}+1$(取等\ $\leftrightarrow i=1$),且11是个质数,所以我们大胆的认为$\alpha=11$。所以需要证明
$10^{2*(k-i)+1}+1\equiv0 \ \textup{mod} \ 11$。
\begin{align*}
\textbf{证明} \quad 10^{2*(k-i)+1}+1&=1+(11-1)^{2*(k-i)+1} \\
&=1+\sum_{j=0}^T C_T^j*11^j*(-1)^{T-j},T=2*(k-i)+1 \\
&=1+C_T^0*11^0*(-1)^T+\sum_{j=1}^T C_T^j*11^j*(-1)^{T-j} \\
&=1+(-1)^T+\sum_{j=1}^T C_T^j*11^j*(-1)^{T-j}
\end{align*}
\begin{align*}
&\because T=2*(k-i)+1\in \textbf{奇数} \quad \therefore (-1)^T=-1 \\
&\Rightarrow 1-1+\sum_{j=1}^T C_T^j*11^j*(-1)^{T-j} \\
&\Rightarrow \sum_{j=1}^T C_T^j*11^j*(-1)^{T-j} \\
&\textbf{显然} \quad \sum_{j=1}^T C_T^j*11^j*(-1)^{T-j} \equiv 0 \ \textup{mod} \ 11
\end{align*}
\qquad综上所述,可得除了非11外的偶数位回文数都不是质数。\textbf{证毕}。
\end{document}